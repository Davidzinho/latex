% \iffalse meta-comment
%<*internal>
\iffalse
%</internal>
%<*readme>
----------------------------------------------------------------
handout --- handout parte da classe utfpr-pg
Author:  (not set)
E-mail:  (not set)
License: Released under the LaTeX Project Public License v1.3c or later
See:     http://www.latex-project.org/lppl.txt
----------------------------------------------------------------

Some text about the class: probably the same as the abstract.
%</readme>
%<*internal>
\fi
\def\nameofplainTeX{plain}
\ifx\fmtname\nameofplainTeX\else
  \expandafter\begingroup
\fi
%</internal>
%<*install>
\input docstrip.tex
\keepsilent
\askforoverwritefalse
\preamble
----------------------------------------------------------------
handout --- handout parte da classe utfpr-pg
Author:  (not set)
E-mail:  (not set)
License: Released under the LaTeX Project Public License v1.3c or later
See:     http://www.latex-project.org/lppl.txt
----------------------------------------------------------------

\endpreamble
\postamble

Copyright (C) 2014 by (not set) <(not set)>

This work may be distributed and/or modified under the
conditions of the LaTeX Project Public License (LPPL), either
version 1.3c of this license or (at your option) any later
version.  The latest version of this license is in the file:

http://www.latex-project.org/lppl.txt

This work is "maintained" (as per LPPL maintenance status) by
(not set).

This work consists of the file handout.dtx and a Makefile.
Running make generates the derived files README.txt, handout.pdf and handout.cls.
Running make inst installs the files in the user's TeX tree.
Running make install installs the files in the local TeX tree.

\endpostamble

\usedir{tex/latex/handout}
\generate{
  \file{\jobname.cls}{\from{\jobname.dtx}{class}}
}
%</install>
%<install>\endbatchfile
%<*internal>
\usedir{source/latex/handout}
\generate{
  \file{\jobname.ins}{\from{\jobname.dtx}{install}}
}
\nopreamble\nopostamble
\usedir{doc/latex/handout}
\generate{
  \file{README.txt}{\from{\jobname.dtx}{readme}}
}
\ifx\fmtname\nameofplainTeX
  \expandafter\endbatchfile
\else
  \expandafter\endgroup
\fi
%</internal>
% \fi
%
% \iffalse
%<*driver>
\ProvidesFile{handout.dtx}
%</driver>
%<class>\NeedsTeXFormat{LaTeX2e}[1999/12/01]
%<class>\ProvidesClass{handout}
%<*class>
    [2014/04/18 v1.00 handout parte da classe utfpr-pg]
%</class>
%<*driver>
\documentclass{ltxdoc}
\usepackage[a4paper,margin=25mm,left=50mm,nohead]{geometry}
\usepackage[utf8]{inputenc}
\usepackage[numbered]{hypdoc}

\EnableCrossrefs
\CodelineIndex
\RecordChanges
\begin{document}
  \DocInput{\jobname.dtx}
\end{document}
%</driver>
% \fi
%
% \GetFileInfo{\jobname.dtx}
% \DoNotIndex{\newcommand,\newenvironment}
%
%\title{\textsf{handout} --- handout parte da classe utfpr-pg\thanks{This file 
%   describes version \fileversion, last revised \filedate.}
%}
%\author{(not set)\thanks{E-mail: (not set)}}
%\date{Released \filedate}
%
%\maketitle
%
%\changes{v1.00}{2014/04/18}{First public release}
%
% \section{Handout}
%
% \begin{abstract}
% ==== Put abstract text here. ====
% \end{abstract}
%
% \subsection{Utilização}
%
% ==== Put descriptive text here. ====
%
% \DescribeMacro{\dummyMacro}
% This macro does nothing.\index{doing nothing|usage} It is merely an
% example.  If this were a real macro, you would put a paragraph here
% describing what the macro is supposed to do, what its mandatory and
% optional arguments are, and so forth.
%
% \DescribeEnv{dummyEnv}
% This environment does nothing.  It is merely an example.
% If this were a real environment, you would put a paragraph here
% describing what the environment is supposed to do, what its
% mandatory and optional arguments are, and so forth.
%
%\StopEventually{^^A
%  \PrintChanges
%  \PrintIndex
%}
%
% \subsection{Implementação}
%
%    \begin{macrocode}
%<*class>
\LoadClass[a4paper,fleqn]{abntex2}
%    \end{macrocode}
% \begin{macro}{\dummyMacro}
% This is a dummy macro.  If it did anything, we'd describe its
% implementation here.
%    \begin{macrocode}
\newcommand{\dummyMacro}{}
%    \end{macrocode}
% \end{macro}
%
% \begin{environment}{dummyEnv}
% This is a dummy environment.  If it did anything, we'd describe its
% implementation here.
%    \begin{macrocode}
\newenvironment{dummyEnv}{%
}{%
%    \end{macrocode}
% \changes{v1.00a}{2014/04/18}{Added a spurious change log entry to
%   show what a change \emph{within} an environment definition looks
%   like.}
% Don't use |%| to introduce a code comment within a |macrocode|
% environment.  Instead, you should typeset all of your comments with
% LaTeX---doing so gives much prettier results.  For comments within a
% macro/environment body, just do an |\end{macrocode}|, include some
% commentary, and do another |\begin{macrocode}|.  It's that simple.
%    \begin{macrocode}
}
%    \end{macrocode}
% \end{environment}
%
%    \begin{macrocode}
\endinput
%</class>
%    \end{macrocode}
%\Finale
