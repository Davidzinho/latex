\documentclass[12pt,openright,oneside,chapter=TITLE,section=TITLE,
	%subsection=TITLE,
	%subsubsection=TITLE,
	brazil]{utfpr-pg}

%\usepackage{cmap}
%\usepackage[T1]{fontenc}
%\usepackage{lastpage}
%\usepackage{indentfirst}
%\usepackage{color}
%\usepackage{graphicx}
%\usepackage{lipsum}

\usepackage[brazilian,hyperpageref]{backref}

% % Configurações do pacote backref
% % Usado sem a opção hyperpageref de backref
% \renewcommand{\backrefpagesname}{Citado na(s) página(s):~}
% % Texto padrão antes do número das páginas
% \renewcommand{\backref}{}
% % Define os textos da citação
% \renewcommand*{\backrefalt}[4]{
% 	\ifcase #1 %
% 		Nenhuma citação no texto.%
% 	\or
% 		Citado na página #2.%
% 	\else
% 		Citado #1 vezes nas páginas #2.%
% 	\fi}%

%  Informações de dados para CAPA e FOLHA DE ROSTO
%\departamento{Departamento Acadêmico de Informática}
\curso{Bacharelado em Ciência da Computação}
\titulo{Modelo Canônico de\\ Trabalho Acadêmico}
\autor{Nome Autor}
\local{Brasil}
\data{2013}
\orientador{Professor Orientador}
\coorientador{Co Orientador}
\tipotrabalho{Tese de Doutorado}
% O preambulo deve conter o tipo do trabalho, o objetivo, 
% o nome da instituição e a área de concentração 
\preambulo{Modelo canônico de trabalho monográfico acadêmico em conformidade com
as normas ABNT apresentado à comunidade de usuários \LaTeX.}

% % informações do PDF
% \makeatletter
% \hypersetup{
%      	%pagebackref=true,
% 		pdftitle={\@title}, 
% 		pdfauthor={\@author},
%     	pdfsubject={\imprimirpreambulo},
% 	    pdfcreator={LaTeX with abnTeX2},
% 		pdfkeywords={abnt}{latex}{abntex}{abntex2}{trabalho acadêmico}, 
% 		colorlinks=true,       		% false: boxed links; true: colored links
%     	linkcolor=blue,          	% color of internal links
%     	citecolor=blue,        		% color of links to bibliography
%     	filecolor=magenta,      		% color of file links
% 		urlcolor=blue,
% 		bookmarksdepth=4
% }
% \makeatother

% O tamanho do parágrafo é dado por:
%\setlength{\parindent}{1.3cm}

% Controle do espaçamento entre um parágrafo e outro:
%\setlength{\parskip}{0.2cm}  % tente também \onelineskip

%\makeindex

\begin{document}
% Retira espaço extra obsoleto entre as frases.
%\frenchspacing 
% \pretextual
\imprimircapa
% % Folha de rosto (o * indica que haverá a ficha bibliográfica)
% \imprimirfolhaderosto*

% \begin{resumo}
%  Segundo a \citeonline[3.1-3.2]{NBR6028:2003}, o resumo deve ressaltar o
%  objetivo, o método, os resultados e as conclusões do documento. A ordem e a extensão
%  destes itens dependem do tipo de resumo (informativo ou indicativo) e do
%  tratamento que cada item recebe no documento original. O resumo deve ser
%  precedido da referência do documento, com exceção do resumo inserido no
%  próprio documento. (\ldots) As palavras-chave devem figurar logo abaixo do
%  resumo, antecedidas da expressão Palavras-chave:, separadas entre si por
%  ponto e finalizadas também por ponto.

%  \vspace{\onelineskip}
    
%  \noindent
%  \textbf{Palavras-chaves}: latex. abntex. editoração de texto.
% \end{resumo}

% \pdfbookmark[0]{\contentsname}{toc}
% \tableofcontents*
% \cleardoublepage

%\pdfbookmark[0]{\listfigurename}{lof}
%\listoffigures*
%\cleardoublepage

\listofquadros

\textual

 \chapter{Introdução}

 Este documento e seu código-fonte são exemplos de referência de uso da classe
 \textsf{abntex2} e do pacote \textsf{abntex2cite}. O documento 
 exemplifica a elaboração de trabalho acadêmico (tese, dissertação e outros do
 gênero) produzido conforme a ABNT NBR 14724:2011 \emph{Informação e documentação
 - Trabalhos acadêmicos - Apresentação}.

\begin{quadro}[h]
  \caption{Oi vovó}
\end{quadro}


% A expressão ``Modelo Canônico'' é utilizada para indicar que \abnTeX\ não é
% modelo específico de nenhuma universidade ou instituição, mas que implementa tão
% somente os requisitos das normas da ABNT. Uma lista completa das normas
% observadas pelo \abnTeX\ é apresentada em \citeonline{abntex2classe}.

% Sinta-se convidado a participar do projeto \abnTeX! Acesse o site do projeto em
% \url{http://abntex2.googlecode.com/}. Também fique livre para conhecer,
% estudar, alterar e redistribuir o trabalho do \abnTeX, desde que os arquivos
% modificados tenham seus nomes alterados e que os créditos sejam dados aos
% autores originais, nos termos da ``The \LaTeX\ Project Public
% License''\footnote{\url{http://www.latex-project.org/lppl.txt}}.

% Encorajamos que sejam realizadas customizações específicas deste exemplo para
% universidades e outras instituições --- como capas, folha de aprovação, etc.
% Porém, recomendamos que ao invés de se alterar diretamente os arquivos do
% \abnTeX, distribua-se arquivos com as respectivas customizações.
% Isso permite que futuras versões do \abnTeX~não se tornem automaticamente
% incompatíveis com as customizações promovidas. Consulte
% \citeonline{abntex2-wiki-como-customizar} par mais informações.

% Este documento deve ser utilizado como complemento dos manuais do \abnTeX\ 
% \cite{abntex2classe,abntex2cite,abntex2cite-alf} e da classe \textsf{memoir}
% \cite{memoir}. 

% Esperamos, sinceramente, que o \abnTeX\ aprimore a qualidade do trabalho que
% você produzirá, de modo que o principal esforço seja concentrado no principal:
% na contribuição científica.

% Equipe \abnTeX   

% Lauro César Araujo

% % ---
% % Finaliza a parte no bookmark do PDF, para que se inicie o bookmark na raiz
% % ---
% \bookmarksetup{startatroot}% 

% \chapter{Conclusão}
% \lipsum[31-33]

% \postextual

% % Referências bibliográficas
\bibliography{dummy.bib}

% %\glossary
% \printindex

\end{document}
