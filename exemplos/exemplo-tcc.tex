\documentclass[12pt,openright,oneside,chapter=TITLE,section=TITLE,
	%subsection=TITLE,
	%subsubsection=TITLE,
	brazil]{utfpr-pg}

%\usepackage{cmap}
%\usepackage[T1]{fontenc}	% pacote já carregado pela classe
%\usepackage{lastpage}
%\usepackage{color}
%\usepackage{graphicx}
%\usepackage{lipsum}

\usepackage[brazilian,hyperpageref]{backref}
%\usepackage[alf]{abntex2cite}	% pacote de citação já carregado pela classe

% % Configurações do pacote backref
% % Usado sem a opção hyperpageref de backref
% \renewcommand{\backrefpagesname}{Citado na(s) página(s):~}
% % Texto padrão antes do número das páginas
% \renewcommand{\backref}{}
% % Define os textos da citação
% \renewcommand*{\backrefalt}[4]{
% 	\ifcase #1 %
% 		Nenhuma citação no texto.%
% 	\or
% 		Citado na página #2.%
% 	\else
% 		Citado #1 vezes nas páginas #2.%
% 	\fi}%

%  Informações de dados para CAPA e FOLHA DE ROSTO
\departamento{Departamento Acadêmico de Informática}
\curso{Bacharelado em Ciência da Computação}
\titulo{Modelo Canônico de\\ Trabalho Acadêmico da UTFPR}
\autor{Nome Autor}
\local{Brasil}
\data{2014}
\orientador{Professor Orientador}
\coorientador{Co Orientador}
\tipotrabalho{Trabalho de Conclusão de Curso}
% O preambulo deve conter o tipo do trabalho, o objetivo, 
% o nome da instituição e a área de concentração
% Ele não é alterado entre trabalhos do mesmo tipo
\preambulo{Trabalho de Conclusão de Curso apresentada como requisito parcial à obtenção do título de Bacharel em Ciência da Computação, do Departamento de Informática da Universidade Tecnológica Federal do Paraná.}

% informações do PDF
\makeatletter
\hypersetup{
     	%pagebackref=true,
		pdftitle={\@title}, 
		pdfauthor={\@author},
    	pdfsubject={\imprimirpreambulo},
		pdfcreator={PdfLaTeX with abnTeX2},
		pdfkeywords={abnt}{latex}{abntex}{abntex2}{trabalho acadêmico}{utfpr}, 
		colorlinks=true,       		% false: boxed links; true: colored links
    	linkcolor=blue,          	% color of internal links
    	citecolor=blue,        		% color of links to bibliography
    	filecolor=magenta,      		% color of file links
		urlcolor=blue,
		bookmarksdepth=4
}
\makeatother

% O tamanho do parágrafo é dado por:
%\setlength{\parindent}{1.3cm}

% Controle do espaçamento entre um parágrafo e outro:
%\setlength{\parskip}{0.2cm}  % tente também \onelineskip

%\makeindex

\begin{document}
% Retira espaço extra obsoleto entre as frases.
\frenchspacing 
\pretextual
\imprimircapa
% Folha de rosto (o * indica que haverá a ficha bibliográfica)
\imprimirfolhaderosto*

\begin{resumo}
Segundo a NBR6028:2003[3.1-3.2], o resumo deve ressaltar o objetivo, o método, os resultados e as conclusões do documento. A ordem e a extensão destes itens dependem do tipo de resumo (informativo ou indicativo) e do tratamento que cada item recebe no documento original. O resumo deve ser precedido da referência do documento, com exceção do resumo inserido no próprio documento. (\ldots) As palavras-chave devem figurar logo abaixo do resumo, antecedidas da expressão Palavras-chave:, separadas entre si por ponto e finalizadas também por ponto.

\vspace{\onelineskip}
\noindent
\textbf{Palavras-chaves}: latex. abntex. trabalho acadêmico. utfpr.
\end{resumo}

\pdfbookmark[0]{\listfigurename}{lof}
\listoffigures*
\cleardoublepage

\pdfbookmark[0]{\listfigurename}{loq}
\listofquadros*
\cleardoublepage

\pdfbookmark[0]{\listtablename}{lot}
\listoftables*
\cleardoublepage


% sumário
\pdfbookmark[0]{\contentsname}{toc}
\tableofcontents*
\cleardoublepage



\textual

\chapter{Introdução}
Este documento é um pequeno exemplo de uso da classe de de trabalhos acadêmicos da UTFPR baseado no \textsf{abntex2} e pacote \textsf{abntex2cite}. Neste exemplo de documento em específico o foco está no trabalho de conclusão de curso produzido conforme~\citeonline{UTFPR:Normas}.

\chapter{Exemplo de Ilustrações}
Abaixo mostra-se um exemplo de quadro:

\begin{quadro}[h]
	\centering
	\renewcommand{\arraystretch}{1.1}
	\begin{tabular}{|l|l|}
		\hline
		\textit{Data Rate}					&	10Mbps		\\\hline
		\textit{Packet Payload}				&	1436 bytes	\\\hline
		\textit{Number of backoff stages \emph{m}}	&	3			\\\hline
		\textit{Minimum Contention Window Size W}	&	16 \textit{slots} \\\hline
		\textit{Control Modulation Scheme}	&	\textit{BPSK} 1/2 \\\hline
		\textit{Propagation Delay}			&	1$\mu$s		\\
		\hline
	\end{tabular}
	% parbox define o tamanho máximo de largura para a legenda
	\parbox{0.55\textwidth}{\caption{Descrição dos parâmetros usados na simulação~(IEEE Standard for Information technology, 2012b).}}
	\label{table:SimulationParameters}
\end{quadro}


% % ---
% % Finaliza a parte no bookmark do PDF, para que se inicie o bookmark na raiz
% % ---
\bookmarksetup{startatroot}% 

\postextual

% Referências bibliográficas
\bibliography{exemplo-tcc}

%\glossary
%\printindex

\end{document}
